% Tutaj znacznik [h] pozwala wymusić rzeby ten obiekt 
% dostępne są równierz inne np t -top albo nawet wymuszenie przez dodatkowy !
% byłoby wtedy [h!]
% Here the [h] tag allows you to force this object
% others are also available, eg t-top or even forcing with an extra !
% would then be [h!]

% Inny przydatny przykład
%  https://reijz.github.io/2016/01/25/pgfplotstable.html

\begin{figure}[h]
 \centering
        \begin{tikzpicture}
        %%%%%%%%%%%%%%%%%%%%%%%%%%%%%%%%%%%%%
%       KONFIGURACJA OSI        
        \begin{axis}[
            width=0.5\textwidth,
            %height=0.5\textwidth,
            %normalsize,
            scale only axis,
            xlabel={\Large{$\frac{\omega \cdot r}{V} \cdot \a$}},
            %xlabel={\Large{$\frac{\omega r}{V}$}},
            %ylabel={$\cdot/ \delta$},
            %ymin=-0.1,
            ymax=1,
            %xmin=0,
            %xmax=6,
            %legend style={font=\small,at={(0.8,0.3)},anchor=north,style={nodes={right}}},
        ]
% Data to plot are in an ordinary file: ler.tex
        \addplot[draw=blue] table[x=CLX, y=CLY] {ler.txt};

        \addplot[draw=red, dashed,
% jakby miały trafić się jakieś niechciane zera 0 to mogę je automatycznie pomijać:
% If there are any unwanted zeros 0, We can skip them automatically:
        y filter/.expression={y==0 ? nan : y},
        ] table[x=CDX, y=CDY] {ler.txt};

% jak mamy dane w pliku to mozemy na ich podstawie policzyc funkcje
% tutaj biorę kolumnę CDX bo taką mam i wykorzystam ją jako x,
% i na jej podstawie liczę sobie np: f(x)=0.1+0.05*x
% IF we have data in the file, we can calculate functions based on them
% here I take the CDX column because that's what I have and I'll use it as x,
% and based on it I calculate e.g.: f(x)=0.1+0.05*x

%                                       konrzystam ze zminnej a = 0.1        
        \addplot[dashed, color=black, smooth]
        table[x=CDX, y expr=\a+0.05*\thisrow{CDX}]  {ler.txt};

% możemy wstawić dowolny mały napisik
% We can insert any small string
%                                       konrzystam ze zminnej b = 11
        \node [] at (axis cs: 2,0.1)	{\small{$B=\b$}};

% możemy wstawić krzywą z funkcji bez żadnego pliku i to nawet jako strzałka:
% We can insert the curve from the function without any file and even as an arrow:
        \addplot [domain=1:4, samples=100, color=green,->]    {-0.04+0.1*\a*(x^2)};

% legendę można ogarnąć w inny sposób bo zamiast za każdym razem wstawiać \addlegendentry{} można wstawić na końcu \legend{} i po przecinku powstawiać opisy i w taki sposób można pominąć którąś serię. Liczy się kolejność wstawiania.
% The legend can be handled in a different way because instead of inserting \addlegendentry{} every time, you can insert \legend{} at the end and put descriptions after a comma, and in this way, you can skip some series. The insertion order matters.        
        \legend{,,My line,Rising line};
        
        \end{axis}
        \end{tikzpicture}
    \caption{Values of the aerodynamic coefficients.}
    \end{figure}